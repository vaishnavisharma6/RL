\documentclass{article}
\usepackage{graphicx}

\title{Reinforcement learning}
\author{Vaishnavi Sharma}
\date{December 2024}

\begin{document}

\maketitle

\section{Multi-armed bandits}
\subsection{20-12-2024}
Some important/interesting points:\\
1. The reward we get after selecting an action $a_{t}$ at time $t$ will not necessarily be the same when we select same action at another time. All we expect to get is some mean reward. \\
2. There are various estimation methods for reward for each arm that can help us in selecting `best' arm at some time step.\\
3. One of them is sample mean. We calculate sample mean for each action that has been taken previously.\\
4. Similarly, there are various action selection methods. One of them: Select an action whose mean reward estimation is highest. Cons: Not much consideration given to exploration/uncertainty.\\
\textbf{5. The factors which influence our balance with exploration/exploitation: Precise values of estimates, uncertainties, and number of remaining steps.}\\

Two simple methods: $\epsilon$ greedy and greedy method.\\
6. Incremental method for reward estimate: To save memory.

\section{Finite Markov Decision Processes}
Some questions:\\
1. Who decides how does environment behave?\\
2. Why randomness in reward is considered once we reach state $s'$? In short, why is reward $r$ not a function of state reached $s'$?\\
3. Art of deciding boundary between agent and environment.

\end{document}