\documentclass{book}
\usepackage{graphicx}

\title{Reinforcement learning lecture notes}
\author{Vaishnavi Sharma, CPS}
\date{January 2025}


\begin{document}

\maketitle
\chapter{Lecture 1: Introduction}
\section{02-01-2025}
\subsection{Essentials}
References: \\
Sutton, Intro\\
Bertsekas, Neuro-Dynamic programming\\
Bertsekas, Optimal control and dynamic programming\\
Same, RL and optimal control\\
SB: Midterm (Feb 15): 20\\
Quiz(Before mid-term, end Jan): 5\\
Final: 50 (Project: 20, Final: 30)\\
Teams code: 17jggiq

\subsection{Introduction}
Learning theory: Supervised\\
\textbf{Reinforcement learning, derives properties of both supervised and unsupervised learning}\\
Unsupervised\\

\subsection{Components}
Environment: State (something in environment that is important to us)\\
Agent: Decision maker\\
RL: Interaction of agent with environment. \\
Goal of agent: Operate in a way (select a sequence of actions) that maximize long term reward.\\
\textbf{Question:} What is long term?

\subsection{Class of problems}
Notation: N: Number of stages of decision making.\\
Cases:\\
Finite horizon problem.\\
1. N is finite.\\
1 (a). N is a deterministic number.\\
1(a) N is finite but random: Episodic or shortest path problem.\\
Long term reward: $E[\sum_{i = 1}^{N} r_{i}| s_{a} = s]$\\

\noindent 2. N is infinite.\\
2 (a) Discounted rewards of the form
$$
lim_{n \rightarrow \infty} E[\sum_{i = 0}^{N} \gamma^{i} r_{i+1}| s_{a} = s]
$$
2(b) Long term average reward\\

\subsection{Key ingredients to an RL problem}
1. State (Environment)\\
2. Action (Agent)\\
3. Rewards\\

\noindent Example: Communication networks (routing problem)\\
Goal: Transmit files as fast as we can.\\
Main Challenge: Formulation of a proper reward.\\
Once we formulate the problem, we can work with any algorithm. E.g: Temporal difference learning, Monte carlo methods and so on.\\
\textbf{To do:} Simulate Routing problem.

\subsection{Exploration vs exploitation tradeoff (interesting problem in any learning theory)}
Why do we need it? \\
1. Limited time\\
2. Long term reward is not revealed by environment (i.e. limited feedback).\\

Another example: Tic-tac-toe game.\\
State space: $(S_{1}, S_{2}, ....., S_{10})$
($S_{10}$  is to denote which player started the game so that next action can be determined)\\
Reward structure: Sparse rewards (0, 1/2, or 1) at the end of the game.\\
\textbf{To do: Simulate the game}

\textbf{To do:} Let's revise Bandits first.

\end{document}